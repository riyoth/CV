\documentclass[10pt]{article}
\usepackage[margin=3cm]{geometry}
\usepackage[latin1]{inputenc} 
\title{\bfseries\Huge Pier-Olivier Cl\'{e}ment}
\author{(514) 703-4335 | pierolivier.clement@gmail.com}
\date{}
\usepackage{array, xcolor}
\definecolor{lightgray}{gray}{0.8}
\newcolumntype{L}{>{\raggedleft}p{0.20\textwidth}}
\newcolumntype{R}{p{0.8\textwidth}}
\newcommand\VRule{\color{lightgray}\vrule width 0.5pt}
\begin{document}
    \maketitle
    \begin{minipage}[ht]{0.48\textwidth}


    \end{minipage}

    \section*{Formation}
    \begin{tabular}{L!{\VRule}R}
        2010-2014&Baccalaur\'eat en g\'enie des technologies de l'information\\
               & \'Ecole de technologie sup\'erieure\\
               & Universit\'e du Qu\'ebec, Montr\'eal\\
    2010    & Dipl\^ome d'\'etudes coll\'egiales en technique de l'informatique : Gestion de r\'eseau\\
            & Coll\`ege \'Edouard-Montpetit, Longueuil\\
    \end{tabular}
     \section*{Exp\'erience professionnelle}
    \begin{tabular}{L!{\VRule}R}
    2014        &{\bf Administrateur Syst\`eme}\\
                &RFID Academia\\
                &configurer les applications selon les besoins du client\\
                &pr\'eparer les rapports \`a l'aide des donn\'ees recueillies\\
                &d\'evelopper des scripts pour simplifier l'administration des diff\'erents syst\`emes\\
                &assister au d\'eveloppement des syst\`emes informatiques\\
                &\\
    2013        &{\bf Int\'egrateur - T\'el\'ephonie IP  }\\
                &SBK Telecom\\
                &installer et configurer un serveur Asterisk\\
                &\'evaluer plusieurs solutions de surveillance\\
                &d\'evelopper un logiciel d'aiguillage\\
                &\\
    2011--2012 & {\bf Int\'egrateur syst\`eme - r\'eseau IP }\\
                &Ericsson\\
                &configurer des commutateurs\\
                &configurer des serveurs d'acc\`es\\
                &\'elaborer des plans d'adressage IP\\
                &r\'ediger un guide de l'administrateur\\
                &\\
    2010--2011   &{\bf Technicien informatique}\\
                &Institut Nazareth et Louis-Braille\\
                &r\'eparer des aides techniques\\
                &configurer des logiciels adapt\'es\\
                &supporter les usagers\\
                &former les nouveaux employ\'es\\
    \end{tabular}
    \section*{Certification}
    \begin{tabular}{L!{\VRule}R}
    2010-2016   &{\bf CCNA Routing and Switching  }\\
                &\\
    2013-2016 & {\bf CCNA S\'ecurit\'e }\\
                &\\
    \end{tabular}
    \section*{Connaissances particuli\`eres}
    \begin{tabular}{L!{\VRule}R}
        Reseau          & configurer un r\'eseau redondant \`a l'aide d'OSPF\\
                        & configurer un VPN pour acc\`es \`a distance\\
                        & configurer de commutateur et routeur Cisco\\
                        & configurer un ASA\\
                        & configurer un routeur sur Linux avec plusieurs VRF\\
                        &\\
        Linux           & configurer un serveur web (Apache et Nginx)\\
                        & configurer un serveur DNS (Bind)\\
                        & configurer un serveur DHCP (ISC-DHCP)\\
                        & configurer des outils de surveillance (Cacti, Nagios)\\
                        & gestion des fichier journaux (syslog et logstash)\\
                        & gestion de serveur centralis\'e avec Puppet\\
                        &\\
        Virtualisation  & migrer des serveurs physiques vers ESXi\\
                        & migrer des serveurs physiques vers KVM\\
                        & administration de serveur avec Archipel et virsh\\
                        & administration de serveur sur Amazon EC2\\
                        &\\
        Scripting       & d\'evelopper un utilitaire pour la gestion des commutateurs\\
                        & d\'evelopper un g\'en\'erateur de rapport\\
                        & d\'evelopper d'une interface web pour un script existant\\
                        & utiliser python pour g\'en\'erer divers fichiers de configuration\\
                        &\\
    \end{tabular}

    \section*{Activit\'e parascolaire}
    \begin{tabular}{L!{\VRule}R}
    2011 --      & {\bf Membre du Lan ETS}\\
                & \'Ecole de technologie sup\'erieure\\
                & maintenir les infrastructures permanentes\\
                & planifier et impl\'ementer le r\'eseau des \'ev\`enements\\
                &\\
    2013--2014  & {\bf CS Games 2014 - VP Infrastructure}\\
                &\'Ecole de technologie sup\'erieure\\
                & configurer et d\'eployer plus de 200 postes de travail\\
                & assurer le bon fonctionnement des comp\'etitions sur les postes de travail\\
                & fournir les services informatiques n\'ecessaires \`a l'\'ev\'enement\\
                &\\


    \end{tabular}
\end{document}
